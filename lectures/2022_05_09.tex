%!TEX root = ../main.tex

\chapter{Software-sensing with Black-Box Methods}
%!TEX root = ../main.tex

\externaldocument{2022_05_02}

In chapter 3 we have seen classical technology of software-sensing based on \acrlong{kf}:
\begin{figure}[H]
    \centering
    \begin{tikzpicture}[node distance=2.5cm,auto,>=latex']
        \node[left] at (0,4) (u) {$u(t)$};
        \node[block] at (2,4) (sys) {$\Sc$};
        \node[block] at (2,2.3) (k) {$\bar{K}$};
        \node[sum] at (4,2.3) (sum) {};
        \node[right] at (6,4) (y) {$y(t)$};
        \node[block, align=center] at (2,1) (model) {model of \\ $\Sc$};
        \node[above] at (2,5) (dist) {disturbances};

        \draw[->] (dist) -- (sys);
        \draw[->] (u) -- (sys);
        \draw[->] (sys) -- (y);
        \draw[<-,red,line width=0.4mm] (sum) -- (4,4) node[pos=0.2] {$+$};
        \draw[->] (sum) -- (k) node[midway, above] {$e(t)$};
        \draw[->,red,line width=0.4mm] (0.5,4) |- (model);
        \draw[->] (k) -- (model);
        \draw[->] (model) -| (sum) 
        	node[pos=0.85] {$-$}
        	node[right, pos=0.7] {$\hat{y}(t|t-1)$};
        \draw[->,red,line width=0.4mm,transform canvas={yshift=-0.2cm}] (model) -- (6,1) node[right] {$\hat{x}(t|t)$};
        \draw[dashed, blue] (0,0) rectangle (5,3) node[right] {$\mathcal{KF}$};
    \end{tikzpicture}
\end{figure}

\textbf{Note} If the \gls{kf} is the asymptotic one (i.e. $K(t) = \bar{K}$), it is a MIMO LTI system. 

Main features of this approach:
\begin{itemize}
    \item A (\acrlong{wb}/physical) model is needed.
    \item No need (in principle) of a training dataset including measurements of the state to be estimated.
    \item It is a feedback estimation algorithm (feedback correction of the model using estimated output error).
    \item Constructive method (non-parametric, no optimization involved).
    \item Can be used (in principle) to estimate states which are impossible to be measured (also at prototyping/training/design stage).
\end{itemize}

Are there other classes of software-sensing techniques? Yes, black-box approaches with \emph{learning}/\emph{training} from data (system identification).
 
In this chapter we see them focusing on the architecture (we do not need new algorithms, just use something we have already studied). We will re-cast the SW-sensing problem into a system identification problem. 

\section{Linear Time Invariant Systems}\label{sec:BB-SW-LTI}

To find the relationship between $u(t) \rightarrow \hat{x}(t|t)$ and $y(t) \rightarrow \hat{x}(t|t)$ we can use a dataset.
Indeed, if we have a dataset
\begin{align*}
    \left\{ u(1), u(2), \ldots, u(N) \right\} \\
    \left\{ y(1), y(2), \ldots, y(N) \right\} \\
    \left\{ x(1), x(2), \ldots, x(N) \right\}
\end{align*}

we can estimate the relationship (i.e. \gls{tf}s) between the inputs ($u(t)$ and $y(t)$) and the output ($x(t)$) without modelling the system and adopting a \gls{bb} approach instead.

\textbf{Note} This is a supervised training approach, thus, only for the training phase, we need measurements of the state to be estimated (using physical sensor that, in \emph{production phase}, will be replaced by the trained SW-sensor).


\paragraph{Model selection} We focus on this family of models:
\[
	\hat{x}(t|t) = S_{ux}(z, \theta) u(t-1) + S_{yx}(z,\theta)y(t)
\]
where we have at least 1-step delay on $u(t)$. The block scheme of this model is: 
\begin{figure}[H]
    \centering
    \begin{tikzpicture}[node distance=2cm,auto,>=latex']
    	\node[block] (z) {$z^{-1}$};
        \node[block, right of=z] (ux) {$S_{ux}(z, \theta)$};
        \node[block, below of=ux] (yx) {$S_{yx}(z, \theta)$};
        \node[left,left of=z] (u) {$u(t)$};
        \node[left, below of=u] (y) {$y(t)$};
        \node[sum,right of=ux,xshift=1cm,yshift=-1cm] (sum) {};
        \node[right, right of=sum] (x) {$\hat{x}(t|t)$};

        \draw[->] (u) -- (z);
        \draw[->] (z) -- (ux);
        \draw[->] (y) -- (yx);
        \draw[->] (ux) -| (sum);
        \draw[->] (yx) -| (sum);
        \draw[->] (sum) -- (x);
    \end{tikzpicture}
\end{figure}

\paragraph{Performance index}
We define the usual performance index as the \emph{sample variance of the estimation error}: 
\[
    J_N(\theta) = \frac{1}{N}\sum_{t=1}^N \left( x(t) - (S_{ux}(z, \theta) u(t-1) + S_{yx}(z,\theta)y(t)) \right)^2
\]

\paragraph{Optimization}
\[
    \hat{\theta}_N = \argmin_\theta J_N(\theta)
\]

We obtain the \acrlong{bb} software sensor $\hat{x}(t|t)$ as: 
\[
	\hat{x}(t|t) = S_{ux}(z, \hat{\theta}_N) u(t-1) + S_{yx}(z,\hat{\theta}_N)y(t)
\]


\textbf{Note} Once the SW-sensor has been designed (trained), we no longer need samples ofokok $x(t)$.

\textbf{Note} The above method is a classic \gls{bb} parametric approach (using \gls{tf}s) but the same can also be done using 4-SID algorithm.


\section{Non-linear Systems}

In case the system is non-linear, we can use the same idea used in the LTI case (see \ref{sec:BB-SW-LTI}), where we replace the asymptotic \gls{kf} with a non-linear SW-sensor (like EKF, described in \ref{subsec:KF_non-lin_ext})

\begin{figure}[H]
    \centering
    \begin{tikzpicture}[node distance=2cm,auto,>=latex']
        \node[block,dashed border,align=center] at (2,4) (n1) {$\Sc_{NL}$};
        \node[block,double border,align=center] at (2,2) (n2) {$f_{NL}$};
        \node[block,dashed border,align=center] at (2,0) (n3) {model of\\ $\Sc_{NL}$};
        \node[sum] at (4,2) (sum) {};
        \node[left] at (0,4) (u) {$u(t)$};
        \node[left] at (6,4) (y) {$y(t)$};

        \draw[dashed, blue] (0,-1) rectangle (5,3) 
        	node[right, align=center] {non-linear\\SW-sensor};

        \draw[->] (u) -- (n1);
        \draw[->] (0.5,4) |- (n3);
        \draw[->] (n2) -- (n3);
        \draw[->] (sum) -- (n2);
        \draw[->] (n1) -| (sum) node[pos=0.9] {+};
        \draw[<-] (sum) |- (n3) node[pos=0.1] {-};
        \draw[->] (n1) -- (y);
        \draw[->,transform canvas={yshift=-0.3cm}] (n3) -- ++(4,0) node[right] {$\hat{x}(t|t)$};
    \end{tikzpicture}
\end{figure}

\begin{rem}[Block scheme notation for non-linear system] 
	Notation used from now on to represent non-linear systems:
	\begin{figure}[H]
	    \centering
	    \begin{tikzpicture}[node distance=2cm,auto,>=latex']
	        \node[block,double border,align=center] (n1) {non-lin.\\\textbf{static}\\system};
	        \node[block,dashed border,align=center, right, right=3cm of n1](n2) {non-lin.\\\textbf{dynamic}\\system};
	    \end{tikzpicture}
	\end{figure}
\end{rem}

% \begin{rem}
%     In \gls{kf} the E.K.F. extension uses the trick of a time-varying linear gain $K(t)$ but the obvious choice is a non-linear gain (static nonlinear function).
% \end{rem}

The content of the box (non-linear SW-sensor) is:

\begin{figure}[H]
    \centering
    \begin{tikzpicture}[node distance=2cm,auto,>=latex']
        \node[block,dashed border,align=center] at (0,0) (n) {non-linear\\dynamic\\time invariant\\system};
        \draw[<-,transform canvas={yshift=0.3cm}] (n) -- ++(-2,0) node[left] {$u(t)$};
        \draw[<-,transform canvas={yshift=-0.3cm}] (n) -- ++(-2,0) node[left] {$y(t)$};
        \draw[->] (n) -- ++(2,0) node[right] {$\hat{x}(t|t)$};
    \end{tikzpicture}
\end{figure}

The problem is again the \gls{bb} identification of a non-linear dynamic system, starting from a measured training dataset.

There are 4 (3+1) different architectures to design the non-linear SW-sensor.

\paragraph{Architecture \#1} Use a \emph{Dynamical Recurrent Neural Network (Dynamical RNN)} in which we update \emph{static neurons} into \emph{dynamic neurons}.

	\begin{figure}[H]
	    \centering
	    \begin{tikzpicture}[node distance=2cm,auto,>=latex']
	        \node[block,dashed border,align=center] at (0,0) (n) {recurrent\\neural network};
	        \draw[<-,transform canvas={yshift=0.3cm}] (n) -- ++(-2,0) node[left] {$u(t)$};
	        \draw[<-,transform canvas={yshift=-0.3cm}] (n) -- ++(-2,0) node[left] {$y(t)$};
	        \draw[->] (n) -- ++(2,0) node[right] {$\hat{x}(t|t)$};
	    \end{tikzpicture}
	\end{figure}

If we zoom into a single neuron: 

\begin{figure}[H]
    \centering
    \begin{minipage}[t]{0.48\textwidth}
        \centering
        \begin{tikzpicture}[node distance=2.5cm,auto,>=latex']
            \node[block] at (0,4) (a1) {$a_1$};
            \node[block] at (0,3) (a2) {$a_2$};
            \node at (0,2) {$\vdots$};
            \node[block] at (0,1) (ah) {$a_h$};
            \node[sum] at (2,2) (sum) {};
            \node[block,ellipse,align=center] at (3.5,2) (nlf) {$f_{NL}$};

            \draw[<-] (a1) -- ++(-1,0);
            \draw[<-] (a2) -- ++(-1,0);
            \draw[<-] (ah) -- ++(-1,0);

            \draw[->] (a1) -- (sum);
            \draw[->] (a2) -- (sum);
            \draw[->] (ah) -- (sum);
            \draw[<-] (sum) -- ++(0,0.5) node[above] {$b$};
            \draw[->] (sum) -- (nlf);
            \draw[->] (nlf) -- ++(1.5,0);
        \end{tikzpicture}
        \caption*{Static neuron (non-linear static system)}
    \end{minipage}
    \begin{minipage}[t]{0.48\textwidth}
        \centering
        \begin{tikzpicture}[node distance=1.5cm,auto,>=latex']
            \node[block] at (0,4) (a1) {$a_1$};
            \node[block] at (0,3) (a2) {$a_2$};
            \node at (0,2) {$\vdots$};
            \node[block] at (0,1) (ah) {$a_h$};
            \node[sum] at (2,2) (sum) {};
            \node[block,ellipse,align=center] at (3.5,2) (nlf) {$f_{NL}$};
            \node[block, below of=nlf] (z) {$z^{-1}$};
            \node[block, below of=sum] (c) {$c$};

            \draw[<-] (a1) -- ++(-1,0);
            \draw[<-] (a2) -- ++(-1,0);
            \draw[<-] (ah) -- ++(-1,0);

            \draw[->] (a1) -- (sum);
            \draw[->] (a2) -- (sum);
            \draw[->] (ah) -- (sum);
            \draw[<-] (sum) -- ++(0,0.5) node[above] {$b$};
            \draw[->] (sum) -- (nlf);
            \draw[->] (nlf) -- ++(1.5,0);

            \draw[->] (4.5,2) |- (z);
            \draw[->] (z) -- (c);
            \draw[->] (c) -- (sum);
        \end{tikzpicture}
        \caption*{Dynamic neuron (non-linear dynamic system)}
    \end{minipage}
\end{figure}
where $f_{NL}$ is a non-linear function (e.g. sigmoid function).

Using an RNN with dynamic neurons is the most general approach but it is also practically seldom used due to stability issues and convergence of training.

\paragraph{Architecture \#2} Split the SW-sensor into a static non-linear system and a dynamic linear system (namely, a non-recursive FIR scheme)

\begin{figure}[H]
    \centering
    \begin{tikzpicture}[node distance=2cm,auto,>=latex']
        \node[block] at (1.5,0) (yn) {$z^{-1}$};
        \node[block] at (1.5,1.5) (y2) {$z^{-1}$};
        \node[block] at (1.5,2.5) (y1) {$z^{-1}$};

        \node[block] at (1.5,4) (un) {$z^{-1}$};
        \node[block] at (1.5,5.5) (u2) {$z^{-1}$};
        \node[block] at (1.5,6.5) (u1) {$z^{-1}$};

        \node[left] at (0,6.5) (u) {$u(t)$};
        \node[left] at (0,3.2) (y) {$y(t)$};

        \node[block,minimum height=7cm,double border,align=center] at (5.25,3.25) (sys) {non-linear\\static\\parametric\\function\\$f_{NL}(\cdot; \theta)$ \vspace{10pt} \\ (e.g. static\\neural\\network)};

        \draw[->] (u) -- (u1);
        \draw[->] (u1) -- (u2);
        \draw[->,dotted] (u2) -- (un);
        \draw[->] (u1) -- (u1-|sys.west) node[pos=0.5] {$u(t-1)$};
        \draw[->] (u2) -- (u2-|sys.west) node[pos=0.5] {$u(t-2)$};
        \draw[->] (un) -- (un-|sys.west) node[pos=0.5] {$u(t-n_u)$};

        \draw[->] (y) -- (y-|sys.west);
        \draw[->] (1.5,3.2) -- (y1);
        \draw[->] (y1) -- (y2);
        \draw[->,dotted] (y2) -- (yn);
        \draw[->] (y1) -- (y1-|sys.west) node[pos=0.5] {$y(t-1)$};
        \draw[->] (y2) -- (y2-|sys.west) node[pos=0.5] {$y(t-2)$};
        \draw[->] (yn) -- (yn-|sys.west) node[pos=0.5] {$y(t-n_y)$};

        \draw[->] (sys) -- ++(2,0) node[right] {$\hat{x}(t|t)$};

        \draw[decoration={brace}, decorate] (3.5,-0.7) node {} -- (0,-0.7);
        \node[align=center,below] at (1.75,-0.9) {linear dynamic\\system};

        \draw[decoration={brace}, decorate] (7,-0.7) node {} -- (3.6,-0.7);
        \node[align=center,below] at (5.25,-0.9) {non-linear static\\system to be\\estimated};
    \end{tikzpicture}
\end{figure}

\begin{rem}[Pros]
	Some pros about this architecture:
	\begin{itemize}
		\item Training (supervised) done only to the non-linear static part of the SW-sensor (much simpler than the estimation of an RNN).
		\item Stability is guaranteed by construction since it is a static FIR architecture.
	\end{itemize}	
\end{rem}

\begin{rem}[Cons]
    In case of a MIMO system with
    \begin{align*}
        m \text{ inputs: } u(t) = \begin{bmatrix}
            u_1(t) \\
            \vdots \\
            u_m(t)
        \end{bmatrix} \quad p \text{ outputs: } y(t) = \begin{bmatrix}
            y_1(t) \\
            \vdots \\
            y_p(t)
        \end{bmatrix} \quad n \text{ states: } x(t) = \begin{bmatrix}
            x_1(t) \\
            \vdots \\
            x_n(t)
        \end{bmatrix}
    \end{align*}

    the estimation problem is the search of the optimal parameter vector $\theta$ for the function
    \[
        f(\cdot; \theta): \RR^{m\times n_u + p \times (n_y + 1)} \rightarrow \RR^n
    \]    
    Therefore, the I/O size of this non-linear parametric function can be very large.
\end{rem}

\paragraph{Architecture \#3} Like architecture \#2, we split the SW-sensor into a static non-linear system and a linear dynamic system but, this time, with a recursive IIR scheme.

\begin{figure}[H]
    \centering
    \begin{tikzpicture}[node distance=2cm,auto,>=latex']
        \node[block] at (1.5,0) (yn) {$z^{-1}$};
        \node[block] at (1.5,1.5) (y2) {$z^{-1}$};
        \node[block] at (1.5,2.5) (y1) {$z^{-1}$};

        \node[block] at (1.5,4) (un) {$z^{-1}$};
        \node[block] at (1.5,5.5) (u2) {$z^{-1}$};
        \node[block] at (1.5,6.5) (u1) {$z^{-1}$};

        \node[block] at (1.5,8) (xn) {$z^{-1}$};
        \node[block] at (1.5,9.5) (x2) {$z^{-1}$};
        \node[block] at (1.5,10.5) (x1) {$z^{-1}$};

        \node[left] at (0,10.5) (x) {$x(t)$};
        \node[left] at (0,6.5) (u) {$u(t)$};
        \node[left] at (0,3.2) (y) {$y(t)$};

        \node[block,minimum height=11cm,minimum width=1.5cm,double border,align=center] at (5.25,5.25) (sys) {non-linear\\static\\parametric\\function\\$f_{NL}(\cdot; \theta)$};

        \draw[->] (u) -- (u1);
        \draw[->] (u1) -- (u2);
        \draw[->,dotted] (u2) -- (un);
        \draw[->] (u1) -- (u1-|sys.west) node[pos=0.5] {$u(t-1)$};
        \draw[->] (u2) -- (u2-|sys.west) node[pos=0.5] {$u(t-2)$};
        \draw[->] (un) -- (un-|sys.west) node[pos=0.5] {$u(t-n_u)$};

        \draw[->] (y) -- (y-|sys.west);
        \draw[->] (1.5,3.2) -- (y1);
        \draw[->] (y1) -- (y2);
        \draw[->,dotted] (y2) -- (yn);
        \draw[->] (y1) -- (y1-|sys.west) node[pos=0.5] {$y(t-1)$};
        \draw[->] (y2) -- (y2-|sys.west) node[pos=0.5] {$y(t-2)$};
        \draw[->] (yn) -- (yn-|sys.west) node[pos=0.5] {$y(t-n_y)$};

        \draw[->] (x) -- (x1);
        \draw[->] (x1) -- (x2);
        \draw[->,dotted] (x2) -- (xn);
        \draw[->] (x1) -- (x1-|sys.west) node[pos=0.5] {$x(t-1)$};
        \draw[->] (x2) -- (x2-|sys.west) node[pos=0.5] {$x(t-2)$};
        \draw[->] (xn) -- (xn-|sys.west) node[pos=0.5] {$x(t-n_x)$};

        \draw[->] (sys) -- ++(2,0) node[right] {$\hat{x}(t|t)$};

    \end{tikzpicture}
    \caption*{Architecture \#3 during training phase}
\end{figure}

\begin{rem}[Pros \& Cons]
	The advantage is that usually in IIR architecture $n_u$ and $n_y$ are much smaller (thanks to recursion): lower computation 
	effort. 

	For the disadvantages we have to notice that only for the training part we use $x(t)$ data from a physical sensor; then, in production, the recursion comes into play since we need to plug-in the output $\hat{x}(t|t)$ in the recursive part: this feedback can provide instability. 

	\begin{figure}[H]
    \centering
    \begin{tikzpicture}[node distance=2cm,auto,>=latex']
        \node[block] at (1.5,0) (yn) {$z^{-1}$};
        \node[block] at (1.5,1.5) (y2) {$z^{-1}$};
        \node[block] at (1.5,2.5) (y1) {$z^{-1}$};

        \node[block] at (1.5,4) (un) {$z^{-1}$};
        \node[block] at (1.5,5.5) (u2) {$z^{-1}$};
        \node[block] at (1.5,6.5) (u1) {$z^{-1}$};

        \node[block] at (1.5,8) (xn) {$z^{-1}$};
        \node[block] at (1.5,9.5) (x2) {$z^{-1}$};
        \node[block] at (1.5,10.5) (x1) {$z^{-1}$};

        \node[left] at (0,10.5) (x) {};
        \node[left] at (0,6.5) (u) {$u(t)$};
        \node[left] at (0,3.2) (y) {$y(t)$};

        \node[block,minimum height=11cm,minimum width=1.5cm,double border,align=center] at (5.7,5.25) (sys) {non-linear\\static\\parametric\\function\\$f_{NL}(\cdot; \theta)$};

        \draw[->] (u) -- (u1);
        \draw[->] (u1) -- (u2);
        \draw[->,dotted] (u2) -- (un);
        \draw[->] (u1) -- (u1-|sys.west) node[pos=0.5] {$u(t-1)$};
        \draw[->] (u2) -- (u2-|sys.west) node[pos=0.5] {$u(t-2)$};
        \draw[->] (un) -- (un-|sys.west) node[pos=0.5] {$u(t-n_u)$};

        \draw[->] (y) -- (y-|sys.west);
        \draw[->] (1.5,3.2) -- (y1);
        \draw[->] (y1) -- (y2);
        \draw[->,dotted] (y2) -- (yn);
        \draw[->] (y1) -- (y1-|sys.west) node[pos=0.5] {$y(t-1)$};
        \draw[->] (y2) -- (y2-|sys.west) node[pos=0.5] {$y(t-2)$};
        \draw[->] (yn) -- (yn-|sys.west) node[pos=0.5] {$y(t-n_y)$};

        \draw[->] (x) -- (x1);
        \draw[->] (x1) -- (x2);
        \draw[->,dotted] (x2) -- (xn);
        \draw[->] (x1) -- (x1-|sys.west) node[pos=0.5] {$\hat{x}(t-1|t-1)$};
        \draw[->] (x2) -- (x2-|sys.west) node[pos=0.5] {$\hat{x}(t-2|t-2)$};
        \draw[->] (xn) -- (xn-|sys.west) node[pos=0.5] {$\hat{x}(t-n_x|t-n_x)$};

        \draw (sys.east) -| ($(x) + (7.5,1.5)$) node[midway, right] {$\hat{x}(t|t)$};

        \draw ($(x) + (7.5,1.5)$) -- ($(x) + (0,1.5)$);

        \draw[->] ($(x) + (0,1.5)$) |- (x1.west) 
        	node[midway, left] {$\hat{x}(t|t)$};

        \draw[dashed, blue] (0.7,7.2) rectangle (4.6,11.3) node[left,above] {recursive part};
    \end{tikzpicture}
    \caption*{Architecture \#3 during production phase}
\end{figure}
\end{rem}

\paragraph{Architecture \#4} Modification with a-priori signal processing of architectures \#1, \#2 and \#3. The idea is to split the SW-sensor in two stages:
\begin{itemize}
	\item first stage: from $u(t)$ and $y(t)$, $h$ \emph{regressors} $r_i(t)$ are produced starting from physical knowledge of the system (where $h$ is much smaller than the number of $u(t)$ and $y(t)$ signals)
	\item second stage: SW-sensor (could be both linear or non-linear and both static or dynamic system) to be firstly trained and then used in production.
\end{itemize}


\begin{figure}[H]
    \centering
    \begin{tikzpicture}[node distance=2cm,auto,>=latex']
        \node[block, dashed border, minimum width=1.5cm, minimum height=3cm, align=center] at (0,0) (sys) {pre-processing\\filter};
        \node[block, dashed border, minimum height=3cm] at (4,0) (f) {$f(\cdot;\theta)$};

        \draw[<-,transform canvas={yshift=0.5cm}] (sys) -- ++(-2cm,0) node[left] {$u(t)$};
        \draw[<-,transform canvas={yshift=-0.5cm}] (sys) -- ++(-2cm,0) node[left] {$y(t)$};

        \draw[->,transform canvas={yshift=1.2cm}] (sys) -- (f) node[pos=0.5] {$r_1(t)$};
        \draw[->,transform canvas={yshift=0.6cm}] (sys) -- (f) node[pos=0.5] {$r_2(t)$};
        \draw[->,transform canvas={yshift=-1.2cm}] (sys) -- (f) node[pos=0.5] {$r_h(t)$};
        \node at (2,0) {$\vdots$};
        \draw[->] (f) -- ++(2cm,0) node[right] {$\hat{x}(t|t)$};
    \end{tikzpicture}
\end{figure}


The idea is to facilitate the estimation of $f(\cdot; \theta)$ by presenting at its input a smaller and more meaningful set of signals (regressors). In this way the \gls{bb} model identification is much simpler.

\paragraph{Conclusions} In case of \gls{bb} SW-sensing with non-linear systems the problem can be quite complex.
Using \emph{brute-force} approach (1 dynamic neural network) is usually doomed to failure.
The best is to gain some insight into the system and build some \emph{smart} regressors before black-box map.

\section{Comparison between \gls{kf} and \gls{bb} software sensing}

\begin{table}[htpb]
    \centering
    \bgroup
    \def\arraystretch{1.5}
    \begin{tabular}{l|c|c}
        & \textbf{\gls{kf}} & \textbf{\gls{bb}} \\
        \hline\hline 
        Need of (\gls{wb}) physical model of the system & \color{red} Yes & \color{green} No \\ \hline 
        Need of a training dataset & \color{green} No {\color{black} \footnote{In practice some tuning through data is needed.}} & \color{red} Yes \\ \hline 
        Interpretability of the obtained SW-sensor & \color{green} Yes & \color{red} No \\ \hline 
        Easy retuning for a similar (different) system & \color{green} Yes & \color{red} No \\ \hline 
        Accuracy of the obtained SW-sensor & \color{green} Good & \color{green} Very Good \\ \hline 
        Can be used also in case of un-measurable states & \color{green} Yes & \color{red} No \\ \hline\hline 
    \end{tabular}
    \egroup
\end{table}
\FloatBarrier

\begin{exa}[Example \ref{ex:KF_full-proc} continued]
    Model (key equation) of the system:
    \[
        M\ddot{z} = -c(t)(\dot{z}-\dot{z}_d) - K(z-z_d)
    \]

    Measurable input $\ddot{z}$ with an accelerometer, $z-z_d$ measurable output with elongation sensor.
    We want to estimate $\dot{z}$.

    The change is $c(t)$ is a semi-active suspension, can be electronically changed (control variable).

    We can solve the problem with \gls{kf} or we can make an experiment and collect training data:
    \begin{align*}
        c(t)        : & \left\{ c(1), c(2), \cdots, c(N) \right\} \\
        z(t)-z_d(t) : & \left\{ z(1)-z_d(1), z(2)-z_d(2), \cdots, z(N)-z_d(N) \right\} \\
        \ddot{z}(t) : & \left\{ \ddot{z}(1), \ddot{z}(2), \cdots, \ddot{z}(N) \right\} \\
        \dot{z}(t)  : & \left\{ \dot{z}(1), \dot{z}(2), \cdots, \dot{z}(N) \right\} \text{ (just for training)} \\
    \end{align*}

    % this part has been done the 18/05/2020

    Back to the main equation:
    \[
        M\ddot{z} = -K(z-z_d)-C(t)(\dot{z}-\dot{z}_d)
    \]
    \[
        \underbrace{\dot{z}}_{\text{to be estim.}} =
        -\frac{K}{M} \underbrace{\int (z-z_d)dt}_{r_1(t)}
        -\frac{1}{M} \underbrace{\int C(t)(\dot{z}-\dot{z}_d)dt}_{r_2(t)}
    \]

    We also consider this equation
    \[
        \dot{z}_d = \underbrace{\int \ddot{z}_d dt}_{r_3(t)}
    \]

    $r_1(t)$ and $r_2(t)$ are the primary regressors, directly linked to $\dot{z}(t)$. $r_3(t)$ is a secondary regressor, it can help $r_1(t)$.

    Since these regressors are obtained by integration to avoid drifting (by DC components of noise integration) we have to high-pass the inputs with high-pass filters $\left(\frac{z-1}{z-a}\right)$.

    \paragraph{Full filtering scheme} \phantom{lol}
    \begin{figure}[H]
        \centering
        \begin{tikzpicture}[node distance=2cm,auto,>=latex']
            \draw[block, dashed border] (0.5,-0.5) rectangle ++(5.5,6);
            \node[block, double border, minimum width=1.5cm, minimum height=6cm] at (8,2.5) (f) {$f(\cdot, \theta)$};
            \node[left] at (0,0) (c) {$c(t)$};
            \node[left] at (0,3) (d) {$z-z_d$};
            \node[left] at (0,5) (z) {$\ddot{z}$};
            \node[sum] at (2,0) (mult) {$\times$};
            \node[block] at (2,1.5) (d1) {$\frac{z-1}{z}$};
            \node[block] at (3.5,0) (d2) {$\frac{z-1}{z-a}$};
            \node[block] at (5,0) (d3) {$\frac{1}{z-1}$};
            \node[block] at (3.5,3) (d4) {$\frac{z-1}{z-a}$};
            \node[block] at (5,3) (d5) {$\frac{1}{z-1}$};
            \node[block] at (3.5,5) (d6) {$\frac{z-1}{z-a}$};
            \node[block] at (5,5) (d7) {$\frac{1}{z-1}$};

            \node[below] at (3,-0.7) {regressors building block};

            \draw[->] (c) -- (mult);
            \draw[->] (d1) -- (mult);
            \draw[->] (mult) -- (d2);
            \draw[->] (d2) -- (d3);
            \draw[->] (z) -- (d6);
            \draw[->] (d6) -- (d7);
            \draw[->] (d) -- (d4);
            \draw[->] (d4) -- (d5);
            \draw[->] (d) -| (d1);

            \draw[->] (d7) -- (d7-|f.west) node[pos=0.7] {$r_3(t)$};
            \draw[->] (d5) -- (d5-|f.west) node[pos=0.7] {$r_1(t)$};
            \draw[->] (d3) -- (d3-|f.west) node[pos=0.7] {$r_2(t)$};
            \draw[->] (f) -- (9.5,2.5) node[right] {$\hat{\dot{z}}$};
        \end{tikzpicture}
    \end{figure}
\end{exa}


\chapter{Gray-Box System Identification}
Content of chapter 5 (short)

\chapter{Minimum Variance Control}
Minimum Variance Control (MVC) is about design and analysis of feedback systems, it is not about system identification nor software sensing.

Why we dedicate a chapter on \emph{control}?
\begin{itemize}
    \item Control design is the main motivation to system identification and software sensing.
    \item MVC is based on \emph{mathematics} of system identification and software sensing (prediction theory).
    \item MVC can be considered as a general tool of stochastic optimization of feedback systems.
\end{itemize}

\section{MVC System}

\paragraph{Setup of the problem} Consider a generic \gls{armax} model

\begin{recall}[\gls{armax} system]
	\[
	    y(t) = \frac{B(z)}{A(z)}u(t-k) + \frac{C(z)}{A(z)}e(t) \qquad e(t) \sim WN(0, \lambda^2)
	\]
	\begin{align*}
	    B(z) &= b_0 + b_1z^{-1} + \dots + b_pz^{-p} \\
	    A(z) &= 1   + a_1z^{-1} + \dots + a_mz^{-m} \\
	    C(z) &= 1   + c_1z^{-1} + \dots + c_nz^{-n}
	\end{align*}

	\textbf{Note}: The input is $u(t)$ and the output is $y(t)$. The noise $e(t)$ is a non-measurable input that models the uncertainty of the system.  
\end{recall}

Assumptions:
\begin{itemize}
    \item $\frac{C(z)}{A(z)}$ is in \emph{canonical form}
    \item $b_0\ne 0$ (this implies that $k \ge 0$ is the actual delay)
    \item $\frac{B(z)}{A(z)}$ is \emph{minimum phase}
\end{itemize}

\begin{recall}[Transfer function in canonical form]
    A transfer function $W(z) = \frac{C(z)}{A(z)}$ is in \emph{canonical form} if 
    \begin{enumerate}
        \item $C(z)$ and $A(z)$ are monic (i.e. the coefficients of the maximum degree terms of $C(z)$ and $A(z)$ are equal to 1);
        \item $C(z)$ and $A(z)$ have null relative degree;
        \item $C(z)$ and $A(z)$ are coprime (i.e. they have no common factors);
        \item[4a.]the poles of $W(z)$ are such that $|z| < 1$;
        \item[4b.]the zeros of $W(z)$ are such that $|z| < 1$. (more stringent condition than the one we saw in MIDA1)
    \end{enumerate}
\end{recall}

\begin{remark}
    $\frac{B(z)}{A(z)}$ is said to be \emph{minimum phase} if all the roots of $B(z)$ are strictly inside the unit circle.

    What does it means in practice? 

    \begin{figure}[H]
        \centering
        \begin{tikzpicture}[node distance=2cm,auto,>=latex']
            \draw[->] (0,4) -- (0,5) node[left] {$u(t)$};
            \draw[->] (0,4) -- (5,4) node[below] {$t$};
            \draw[->] (0,0) -- (0,3) node[left] {$y(t)$};
            \draw[->] (0,0) -- (5,0) node[below] {$t$};

            \draw[line width=0.4mm] (0,4) -- (1,4) -- (1,4.5) -- (5,4.5);
            \draw[dotted] (1,0) -- (1,4);
            \draw[dotted] (0,2.5) -- (5,2.5) node[right] {\footnotesize steady state};

            \draw[domain=1:4.5,smooth,variable=\x,green,samples=70] plot ({\x},{2.5-2.5*(1-(\x-1)/3.5)^5});
            \draw[domain=1:4.5,smooth,variable=\x,green,samples=70] plot ({\x},{2.5-2.5*(1-(\x-1)/3.5)^5+sin(\x*180)/\x});
            \draw[domain=1:4.5,smooth,variable=\x,red,samples=70] plot ({\x},{2.5-2.5*(1-(\x-1)/3.5)^5+3*sin(\x*180)/(\x^2)});

            \node[red,right] at (1.5,-0.5) {\footnotesize non- minimum phase};
            \node[green,right] at (2,1.5) {\footnotesize minimum phase};
        \end{tikzpicture}
    \end{figure}

    At the very beginning the response of a \textcolor{red}{non-minimum phase} system goes to the opposite direction w.r.t. the final value. Intuitively it's very difficult to control non-minimum phase systems: you can take the wrong decision if you react immediately.

    Also for human it's difficult, for example \emph{steer to roll} dynamics in a bicycle: if you want to steer left, you must first steer a little to the right and then turn left.

    Design of controller for non-minimum phase is difficult and requires special design techniques (no MVC but \emph{generalized MVC (GMVC)}, described in \ref{subsec:GMVC}).
\end{remark}

Some additional (small) technical assumptions:
\begin{itemize}
    \item $y^0(t)$ and $e(t)$ are not correlated (usually fulfilled).
    \item We assume (worst case) that $y^0(t)$ is known only up to time $t$ (present time): we have no preview of the future desired $y^0(t)$ (i.e. $y^0(t)$ is totally unpredictable or $\hat{y^0}(t+k|t) = y^0(t)$).
\end{itemize}

\paragraph{Goal of the problem} 
The problem we wish to solve is optimal tracking of the desired behavior of output (which is the classical goal of control systems):
\begin{figure}[H]
    \centering
    \begin{tikzpicture}[node distance=2cm,auto,>=latex']
        \node[block,ellipse,align=center] at (-1,0) (cont) {controller\\algorithm};
        \node[block] at (2.5,0) (ba) {$\frac{B(z)}{A(z)}z^{-k}$};
        \node[block] at (4,1.5) (ca) {$\frac{C(z)}{A(z)}$};
        \node[sum] at (4,0) (sum) {};

        \draw[dotted] (0.8,-1) rectangle (5,3.2) node[right] {system to be controlled};
        \draw[->] (cont) -- (ba) node[pos=0.7] {$u(t)$};
        \draw[->] (ba) -- (sum);
        \draw[->] (ca) -- (sum);
        \draw[<-] (cont) -- ++(-2,0) node[left] {$y^0(t)$};
        \draw[<-] (ca) -- ++(0,1) node[above] {$e(t)$};
        \draw[->] (sum) -- ++(2,0) node[right] {$y(t)$};
        \draw[->] (5.5,0) -- (5.5,-1.5) -- (-1,-1.5) -- (cont);
    \end{tikzpicture}
\end{figure}
where $y^0(t)$ os the desired output value, called \emph{reference}. 

In a more formal way MVC is an optimization control problem that tries to find $u(t)$ that minimize this performance index:
\[
    J = E\left[ (y(t) - y^0(t))^2 \right]
\]

where $J$ is the variance of the tracking error: that's why it's called Minimum Variance Control.

\paragraph{Solution of the problem}
The main trick is to split $y(t)$ into $\hat{y}(t|t-k)$, the \emph{predictor}, and $\epsilon(t)$, the \emph{prediction error}.

Now we can write $J$ as 
\begin{align*}
	J =& \E \left[ \left( \hat{y}(t|t-k) + \epsilon(t) - y^0(t) \right)^2 \right] \\
	  =& \E \left[ \left( (\hat{y}(t|t-k) - y^0(t)) + \epsilon(t) \right)^2 \right] \\
	  =& \E \left[ \left( (\hat{y}(t|t-k) - y^0(t) \right)^2 \right] + \E [\epsilon(t)^2 ] + 2 \E \left[ \epsilon(t) \left( \hat{y}(t|t-k) + y^0(t) \right) \right] 
\end{align*}

where we note that

\begin{itemize}
	\item $\E [\epsilon(t)^2 ]$ doesn't depend on $u(t)$: it's not subject to minimization, since it's just a constant number.
	\item $\E [ \epsilon(t) y^0(t) ] = 0$, since $e(t) = f(e(t), e(t-1), \dots) $ and, by assumption, $y^0 \perp e$.
	\item $\E [ \epsilon(t) \hat{y}(t|t-k) ] = 0$, since by construction $\epsilon(t) \perp \hat{y}(t|t-k)$ (otherwise $\hat{y}(t|t-k)$ wouldn't be the optimal predictor).
\end{itemize}

Therefore, minimizing $J$ is equivalent to minimizing  $\E \left[ \left( (\hat{y}(t|t-k) - y^0(t) \right)^2 \right]$, which is minimized if

\begin{align}\label{eq:min_J_condition}
	\hat{y}(t|t-k) = y^0(t)
\end{align}

Now we have to compute $\hat{y}(t|t-k)$ and impose $\hat{y}(t+k|t) = y^0(t+k)$ (obtained by shifting the equation \ref{eq:min_J_condition} using the $z$ operator).

However, by assumption, at time $t$ we don't know $y^0(t+k)$: the best we can do is to replace it with the last available value of $y^0$, that is $y^0(t)$.

Therefore, the condition \ref{eq:min_J_condition} becomes 

\begin{align}\label{eq:new_min_J_condition}
	\hat{y}(t+k|t) = y^0(t)
\end{align}

\begin{recall}[$k$-step predictor of an \gls{armax} system]

	\[
		\hat{y}(t+k|t) = \frac{B(z) E(z)}{C(z)} u(t) + \frac{\tilde{R}(z)}{C(z)} y(t)
	\] 

	where $E(z)$ is the $result$ and $R(z) = \tilde{R}(z) z^{-k}$ is the $residual$ of the $k$-step long division between $C(z)$ and $A(z)$.

\end{recall}

By plugging in the $k$-step predictor of an \gls{armax} system formula in the equation \ref{eq:new_min_J_condition} we obtain

\[
	\frac{B(z) E(z)}{C(z)} u(t) + \frac{\tilde{R}(z)}{C(z)} y(t) = y^0(t)
\]

and by making $u(t)$ explicit we obtain the \emph{General Formula of MVC}:

\begin{align}\label{eq:MVC_general-formula}
	u(t) = \frac{1}{B(z)E(z)} \left( C(z) y^0(t) - \tilde{R}(z) y(t) \right)
\end{align}

The block scheme representation of the MVC system is

\begin{figure}[H]
    \centering
    \begin{tikzpicture}[node distance=2cm,auto,>=latex']
        \node[sum] at (0,0) (sum) {};
        \node[block, left=1cm of sum] (c1) {$C$}; 
        \node[block] at (1.5,0) (b1) {$\frac{1}{B(z) E(z)}$};
        \node[block] at (5,0) (b2) {$\frac{B(z)}{A(z)} z^{-k}$};
        \node[block] at (3,-1.5) (b3) {$\tilde{R}(z)$};
        \node[block] at (7,1.5) (b4) {$\frac{C(z)}{A(z)}$};
        \node[sum] at (7,0) (sum2) {};

        \draw[<-] (c1) --++ (-1,0) {} node[left] (in) {$y^0(t)$};
        \draw[->] (sum) -- (b1);
        \draw[->] (b1) -- (b2) node[pos=0.5] {$u(t)$};
        \draw[->] (b3) -| (sum) node[pos=0.9] {$-$};
        \draw[->] (c1) -- (sum)  node[above, near end] {$+$};
        \draw[->] (sum2) -- ++(1.5,0) node[right] {$y(t)$};
        \draw[->] (b2) -- (sum2);
        \draw[->] (b4) -- (sum2);
        \draw[<-] (b4) -- ++(0,1) node[above] (noise_in) {$e(t)$};
        \draw[->] (8,0) |- (b3);

        \draw[dashed, blue] ($(in) + (0.7, -2.5)$) rectangle ($(b2) + (-1.2, 1)$) node[above=0.6cm of c1] {\text{controller}};
        \draw[dashed] ($(b2) + (-1, -0.6)$) rectangle ($(noise_in) + (0.7, 0.3)$) node[right] {\text{system}};
    \end{tikzpicture}

    \caption{MVC System}
    \label{fig:MVC_sys}
\end{figure}

\section{Analysis of the MVC System}

\subsection{Stability of the MVC System}\label{subsec:MVC_stability}

\begin{recall}
    For stability let's recall a result of \emph{negative feedback system}:
    \begin{figure}[H]
        \centering
        \begin{tikzpicture}[node distance=2cm,auto,>=latex']
            \node[block] at (0,1.5) (f1) {$F_1(t)$};
            \node[block] at (0,0.5) (f2) {$F_2(t)$};
            \node[sum] at (-1.5,1.5) (sum) {};

            \draw[<-] (sum) -- ++(-1,0) node[pos=0.2, above] {$+$};
            \draw[->] (f2) -| (sum) node[pos=0.9] {$-$};
            \draw[->] (sum) -- (f1);
            \draw[->] (f1) -- ++(2,0);
            \draw[->] (1.5,1.5) |- (f2);
        \end{tikzpicture}
    \end{figure}

    To check the closed-loop stability:
    \begin{itemize}
        \item compute the \emph{loop-function} $L(z) = F_1(z) F_2(z)$ (\textbf{remember}: do not simplify!)
        \item build the \emph{characteristic polynomial} $\chi(z) = L_N(z) + L_D(z)$ (sum of the numerator and the denominator of $L(z)$)
        \item find the roots of $\chi(z)$: closed loop system is asymptotically stable if and only if all the roots of $\chi(z)$ are strictly inside the unit circle
    \end{itemize}
\end{recall}

Check the stability of the MVC system represented in figure \ref{fig:MVC_sys}

\[
    L(z) = \frac{1}{B(z)E(z)}\cdot \frac{z^{-k}B(z)}{A(z)}\cdot\tilde{R}(z)
\]
\begin{align*}
    \chi(z) &= z^{-k}B(z)\tilde{R}(z) + B(z)E(z)A(z) \\
    &= B(z) \underbrace{\left( z^{-k}\tilde{R}(z)+E(z)A(z) \right)}_{C(z)}  \\
    &= B(z)C(z)
\end{align*}

where we used the following result of the long division between $C(z)$ and $A(z)$:

\[ \frac{C(z)}{A(z)} = E(z) + \frac{z^{-k} \tilde{R}(z)}{A(z)} \]

The MVC system is always guaranteed asymptotically stable, since the roots of $\chi(z)$ are the roots of $B(z)$ and $C(z)$ and:
\begin{itemize}
    \item all roots of $B(z)$ are strictly inside the unit circle (thanks to minimum phase assumption on $\frac{B(z)}{A(z)}$)
    \item all roots of $C(z)$ are inside the unit circle (thanks to the assumption of canonical representation of $\frac{C(z)}{A(z)}$)
\end{itemize}

\subsection{Performance of the MVC System}\label{subsec:MVC_performance}

The system can be rewritten as follows considering the two inputs: $y^0(t)$ and $e(t)$ (non-measurable input)
\[ 
	y(t) = F_{y^0y}(z) y^0(t) + F_{ey}(z) e(t)
\]

\begin{recall}
    The transfer function from the input to the output of a \emph{negative feedback system} can be computed with:
    \begin{figure}[H]
        \centering
        \begin{tikzpicture}[node distance=2cm,auto,>=latex']
            \node[block] at (0,1.5) (f1) {$F_1(t)$};
            \node[block] at (0,0.5) (f2) {$F_2(t)$};
            \node[sum] at (-1.5,1.5) (sum) {};

            \draw[<-] (sum) -- ++(-1,0) node[left] {$u(t)$} node[pos=0.2, above] {$+$} ;
            \draw[->] (f2) -| (sum) node[pos=0.9] {$-$};
            \draw[->] (sum) -- (f1);
            \draw[->] (f1) -- ++(2,0) node[right] {$y(t)$};
            \draw[->] (1.5,1.5) |- (f2);
        \end{tikzpicture}
    \end{figure}

    \[ 
    	F(z) = \frac{F_1(z)}{1 + F_1(z) F_2(z)} \qquad y(t) = F(z) u(t)
    \]
    where we recall that $F_1(z) F_2(z)$ is the loop-function $L(z)$
 	and $F_1(z)$ is the direct line from the input to the output.
\end{recall}

Therefore, the \gls{tf} from $y^0(t)$ to $y(t)$ is
\begin{align*}
	F_{y^0y}(z) =& \frac{C(z) \cdot \frac{1}{B(z) E(z)} \cdot \frac{z^{-k} B(z)}{A(z)}}{1 + \underbrace{\frac{1}{B(z)E(z)}\cdot \frac{z^{-k}B(z)}{A(z)}\cdot\tilde{R}(z)}_{L(z)}} \\
	=& \dots \\
	=& z^{-k}
\end{align*}
where we remind that $L(z)$ has been already computed for the stability check (\ref{subsec:MVC_stability}). 

Similarly, the \gls{tf} from $e(t)$ to $y(t)$ is 
\begin{align*}
	F_{ey}(z) =& \frac{\frac{C(z)}{A(z)}}{1 + \underbrace{\frac{1}{B(z)E(z)}\cdot \frac{z^{-k}B(z)}{A(z)}\cdot\tilde{R}(z)}_{L(z)}} \\
	=& \dots \\
	=& E(z)
\end{align*}

Thus we can say that

\begin{align*}
	y(t) =& F_{y^0y}(z) y^0(t) + F_{ey}(z) e(t) \\
		 =& z^{-k} y^0(t) + E(z) e(t) \\
         =& y^0(t-k) + E(z) e(t)
\end{align*}
 
which is the very simple closed-loop relationship between input and output in a MVC system.

% \begin{remark}
%     There are 2 sub-classes of control problems:
%     \begin{itemize}
%         \item When $y^0(t)$ is constant or step-wise (regulation problem)
%         \item When $y^0(t)$ is varying (tracking problem)
%     \end{itemize}

%     \begin{figure}[H]
%         \centering
%         \begin{minipage}[t]{0.48\textwidth}
%             \centering
%             \begin{tikzpicture}[
%                 node distance=2cm,auto,>=latex',
%                 declare function={
%                     f(\x) =  (\x < 0.5) * 1 +
%                              (\x >= 0.5) * (\x < 2) * 2 +
%                              (\x >= 2) * (\x < 3) * 3 +
%                              (\x >= 3) * (\x < 4) * 1.5 +
%                              (\x >= 4) * 1;
%                     f2(\x) = (f(\x-0.5) + (f(\x) - f(\x-0.5)) / 720 +
%                              f(\x-0.4) + (f(\x) - f(\x-0.4)) / 120 +
%                              f(\x-0.3) + (f(\x) - f(\x-0.3)) / 24 +
%                              f(\x-0.2) + (f(\x) - f(\x-0.2)) / 6 +
%                              f(\x-0.1) + (f(\x) - f(\x-0.1)) / 2) / 5 +
%                              rand/8;
%                 }
%             ]
%                 \draw[->] (0,0) -- (0,3) node[above] {$y(t)$};
%                 \draw[->] (0,0) -- (5,0) node[below] {$t$};
%                 \draw[domain=0:5,variable=\x,blue,samples=100] plot ({\x},{f(\x)});
%                 \draw[domain=0:5,variable=\x,red,smooth,samples=100] plot ({\x},{f2(\x)});
%             \end{tikzpicture}
%             \caption*{Regulation problem}
%         \end{minipage}
%         \begin{minipage}[t]{0.48\textwidth}
%             \centering
%             \begin{tikzpicture}[
%                 node distance=2cm,auto,>=latex',
%                 declare function={
%                     f(\x) =  (sin(\x*180)/2+sin(\x*270)/2)+1.5;
%                     f2(\x) = (f(\x-0.3) + (f(\x) - f(\x-0.3)) / 24 +
%                              f(\x-0.2) + (f(\x) - f(\x-0.2)) / 6 +
%                              f(\x-0.1) + (f(\x) - f(\x-0.1)) / 2) / 3 +
%                              rand/8;
%                 }
%             ]
%                 \draw[->] (0,0) -- (0,3) node[above] {$y(t)$};
%                 \draw[->] (0,0) -- (5,0) node[below] {$t$};
%                 \draw[domain=0:5,variable=\x,blue,samples=70] plot ({\x},{f(\x)});
%                 \draw[domain=0:5,variable=\x,red,smooth,samples=50] plot ({\x},{f2(\x)});
%             \end{tikzpicture}
%             \caption*{Tracking problem}
%         \end{minipage}
%     \end{figure}
% \end{remark}

% Bottom-up way of presenting M.V.C.

% \paragraph{Simplified problem \#1}
% \[
%     S: y(t) = ay(t-1) + b_0u(t-1) + b_1u(t-2) \qquad y(t) = \frac{b_0+b_1z^{-1}}{1-az^{-1}}u(t-1)
% \]

% We assume that $y^0(t)=\overline{y}^0$ (regulation problem) and the system is noise-free.
% \begin{itemize}
%     \item $b_0\ne 0$
%     \item Root of numerator must be inside the unit circle
% \end{itemize}

% To design the minimum variance controller we must minimize the performance index:
% \[
%     J = E\left[ (y(t) - y^0(t))^2 \right]
% \]
% There is no noise so we can remove the expected value
% \begin{align*}
%     J &= \left( y(t) - y^0(t) \right)^2 = \left( y(t) - \overline{y}^0 \right)^2 = \left( ay(t-1)+b_0u(t-1)+b_1u(t-2) - \overline{y}^0 \right)^2 = \\
%     &= \left( ay(t) + b_0u(t) + b_1u(t-1)-\overline{y}^0 \right)^2 \\
%     \frac{\partial J}{\partial u(t)} &= 2\left( ay(t)+b_0u(t)+b_1u(t-1)-\overline{y}^0 \right)\left(b_0\right)
% \end{align*}

% Why the derivative is just $b_0$? We are at present time $t$ and at time $t$ the control algorithm must take a decision on the value of $u(t)$.
% At time $t$, $y(t)$, $y(t-1)$, \dots, $u(t-1)$, $u(t-2)$, \dots{} are no longer variables but numbers.

% \[
%     \frac{\partial J}{\partial u(t)} = 0 \qquad ay(t)+b_0u(t)+b_1u(t-1)-\overline{y}^0 = 0 \\
% \]
% \[
%     u(t) = \left( \overline{y}^0  - ay(t)\right)\frac{1}{b_0+b_1z^{-1}}
% \]
% \begin{figure}[H]
%     \centering
%     \begin{tikzpicture}[node distance=2cm,auto,>=latex']
%         \node[sum] at (0,0) (sum) {};
%         \node[block] at (1.5,0) (b1) {$\frac{1}{b_0+b_1z^{-1}}$};
%         \node[block] at (5,0) (b2) {$z^{-1}\frac{b_0+b_1z^{-1}}{1-az^{-1}}$};
%         \node[block] at (3,-1.5) (b3) {$a$};

%         \draw[->] (sum) -- (b1);
%         \draw[->] (b1) -- (b2) node[pos=0.5] {$u(t)$};
%         \draw[->] (b3) -| (sum) node[pos=0.9] {-};
%         \draw[<-] (sum) -- ++(-1.5,0) node[left] {$\overline{y}^0$} node[pos=0.2] {+};
%         \draw[->] (b2) -- ++(2.5,0) node[right] {$y(t)$};
%         \draw[->] (6.5,0) |- (b3);
%     \end{tikzpicture}
% \end{figure}

% \paragraph{Simplified problem \#2}
% \[
%     S: y(t) = ay(t-1) + b_0u(t-1) + b_1u(t-2) + e(t) \qquad e(t) \sim WN(0, \lambda^2)
% \]

% The reference variable is $y^0(t)$ (tracking problem).

% The performance index is
% \[
%     J = E\left[ (y(t) - y^0(t))^2 \right]
% \]

% The fundamental trick to solve this problem is to re-write $y(t)$ as:
% \[
%     y(t) = \hat{y}(t|t-1) + \epsilon(t)
% \]

% Since $k=1$ we know that $\epsilon(t) = e(t)$, so $y(t) = \hat{y}(t|t-1)+e(t)$.
% \begin{align*}
%     J &= E\left[ \left(\hat{y}(t|t-1)+e(t) - y^0(t)\right)^2 \right] \\
%     &= E\left[   \left((\hat{y}(t|t-1)-y^0(t)) +e(t)\right)^2 \right] \\
%     &= E\left[ \left(\hat{y}(t|t-1)-y^0(t)\right)^2 \right] + E\left[e(t)^2\right] + \cancel{2E\left[e(t)\left( \hat{y}(t|t-1)-y^0(t) \right)\right]} \\
% \end{align*}

% Notice that
% \[
%     \argmin_{u(t)} \left\{ E\left[ \left(\hat{y}(t|t-1)-y^0(t)\right)^2 \right] + \lambda^2 \right\} = \argmin_{u(t)} \left\{ E\left[ \left(\hat{y}(t|t-1)-y^0(t)\right)^2 \right] \right\}
% \]
% The best result is when $\hat{y}(t|t-1)=y^0(t)$, we can force this relationship.

% Now we must compute the 1-step predictor of the system:
% \[
%     S: y(t) = \frac{b_0+b_1z^{-1}}{1-az^{-1}}u(t-1) + \frac{1}{1-az^{-1}}e(t)
% \]
% Note that this is an $ARMAX(1,0,1+1)=ARX(1,2)$.
% \[
%     k=1 \qquad B(z) = b_0+b_1z^{-1} \qquad A(z)=1-az^{-1} \qquad C(z) = 1
% \]

% General solution for 1-step prediction of ARMAX:
% \[
%     \hat{y}(t|t-1) = \frac{B(z)}{C(z)}u(t-1) + \frac{C(z)-A(z)}{C(z)}y(t)
% \]
% If we apply this formula we obtain:
% \[
%     \hat{y}(t|t-1) = \frac{b_0+b_1z^{-1}}{1}u(t-1) + \frac{1-1+az^{-1}}{1}y(t) = (b_0+b_1z^{-1})u(t-1)+ay(t-1)
% \]

% Now we can impose that $\hat{y}(t|t-1)=y^0(t)$
% \[
%     b_0u(t) + b_1u(t-1) + ay(t) = y^0(t+1) \qquad u(t) = \left( y^0(t+1) - ay(t) \right)\frac{1}{b_0+b_1z^{-1}}
% \]
% But we don't have $y^0(t+1)$, so we use $y^0(t)$.
% \[
%     u(t) = \left( y^0(t) - ay(t) \right)\frac{1}{b_0+b_1z^{-1}}
% \]
% \begin{figure}[H]
%     \centering
%     \begin{tikzpicture}[node distance=2cm,auto,>=latex']
%         \node[sum] at (0,0) (sum) {};
%         \node[block] at (1.5,0) (b1) {$\frac{1}{b_0+b_1z^{-1}}$};
%         \node[block] at (5,0) (b2) {$z^{-1}\frac{b_0+b_1z^{-1}}{1-az^{-1}}$};
%         \node[block] at (3,-1.5) (b3) {$a$};
%         \node[block] at (7,1.5) (b4) {$\frac{1}{1-az^{-1}}$};
%         \node[sum] at (7,0) (sum2) {};

%         \draw[->] (sum) -- (b1);
%         \draw[->] (b1) -- (b2) node[pos=0.5] {$u(t)$};
%         \draw[->] (b3) -| (sum) node[pos=0.9] {-};
%         \draw[<-] (sum) -- ++(-1.5,0) node[left] {$y^0(t)$} node[pos=0.2] {+};
%         \draw[->] (sum2) -- ++(1.5,0) node[right] {$y(t)$};
%         \draw[->] (b2) -- (sum2);
%         \draw[->] (b4) -- (sum2);
%         \draw[<-] (b4) -- ++(0,1) node[above] {$e(t)$};
%         \draw[->] (8,0) |- (b3);
%     \end{tikzpicture}
% \end{figure}

