\chapter{Appendix: Discretization of an analog system}
\newlecture{Sergio Savaresi}{04/06/2020}

\missingfigure{Fig1}

\subsection*{A to D converter}

\missingfigure{Fig2}

\begin{description}
    \item[Time discretization] $\Delta T$ is the sampling time
    \item[Amplitude discretization] Number of levels used for discretization, e.g. \emph{10-bit discretization} uses $2^{10}$ levels of amplitude
\end{description}

An high quality of A/D converter:
\begin{itemize}
    \item Can use small $\Delta T$
    \item High number of levels (16-bits)
\end{itemize}


\subsection*{D to A converter}

\missingfigure{Fig3}

If $\Delta T$ is sufficiently small, the step-wise analog signal is very similar to a smooth analog signal.

\missingfigure{Fig4}

What is the model of the Digital Perspective?

\missingfigure{Fig5}

\begin{itemize}
    \item We make B.B. system identification from measured data: directly estimate a discrete-time model
    \item We have a physical W.B. model (continuous time), we need to discretize it
\end{itemize}

The most used approach is the State-Space Transformation.

\[
    S: \begin{cases}
        \dot{x} = Ax + bu \\
        y = Cx + (Du)
    \end{cases}
    \qquad
    \text{sampling time $\Delta T$}
    \qquad
    S: \begin{cases}
        x(t+1) = Fx(t) + Gu(t) \\
        y(t) = Hu(t) + (Du(t))
    \end{cases}
\]

Transformation formulas:
\begin{align*}
    F &= e^{A\Delta T} \\
    G &= \int_0^{\Delta T} e^{A\delta}B\, d\delta \\
    H &= C \\
    D &= D
\end{align*}

\begin{remark}
    How the poles of the continuous time system are transformed?

    Can be proved that the eigenvalues (poles) follow the \emph{sampling transformation rule}.
    \[
        z = e^{s\Delta T} \qquad \lambda_F = e^{\lambda_A \Delta T}
    \]

    \missingfigure{Fig6}

    How the zeros in of $S$ in continuous time are transformed into zeros in discrete time?

    Unfortunately there is no simple rule like the poles. We can only say:
    \[
        G(s) = \frac{\text{polynomial in $s$ with $h$ zeros}}{\text{polynomial in $s$ with $k$ poles}} \qquad \text{if $G(s)$ is strictly proper: } k > h
    \]
    \[
        G(z) = \frac{\text{polynomial in $z$ with $k-1$ zeros}}{\text{polynomial in $z$ with $k$ poles}} \qquad \text{$G(z)$ with relative degree 1}
    \]

    We have new $k-h-1$ zeros that are generated by the discretization.
    They are called \emph{hidden zeros}.

    Unfortunately these hidden zeros are frequently outside the unit circle, which means that $G(z)$ is not minimum phase even if $G(s)$ is minimum phase.

    We need for instance GMVC to design the control system.
\end{remark}

Another simple discretization technique frequently used is the discretization of time-derivative $\dot{x}$.

\begin{align*}
    \text{\textbf{eulero backward}} &\qquad \dot{x} \approx \frac{x(t)-x(t-1)}{\Delta T} = \frac{x(t)-z^{-1}x(t)}{\Delta T} = \frac{z-1}{z\Delta T} x(t) \\
    \text{\textbf{eulero forward}} &\qquad \dot{x} \approx \frac{x(t+1)-x(t)}{\Delta T} = \frac{zx(t)-x(t)}{\Delta T} = \frac{z-1}{\Delta T} x(t)
\end{align*}

General formula
\[
    \dot{x}(t) = \left[ \frac{z-1}{\Delta T} \frac{1}{\alpha z + (1-\alpha)} \right]x(t) \qquad \text{with } 0 \le \alpha \le 1
\]
\begin{itemize}
    \item if $\alpha = 0$ it's Eulero Forward
    \item if $\alpha = 1$ it's Eulero Backward
    \item if $\alpha = \frac{1}{2}$ it's Tustin method
\end{itemize}

The critical choice is $\Delta T$ (sampling time).
The general intuitive rule is: the smaller $\Delta T$, the better.

\missingfigure{Fig7}

If $\Delta T$ is smaller, $\omega_S$ larger

\missingfigure{Fig8}

\[
    \Delta T \rightarrow f_S = \frac{1}{\Delta T} \qquad \omega_S = \frac{2\pi}{\Delta T} \qquad f_N = \frac{1}{2} f_S \qquad \omega_N = \frac{1}{2} \omega_S
\]

Hidden problems of a too-small $\Delta T$:
\begin{itemize}
    \item Sampling devices (A/D and D/A) cost
    \item Computational cost: update an algorithm every $1 \mu s$ is much heavier than every $1 ms$
    \item Cost of memory (if data logging is needed)
    \item Numerical precision \emph{cost} (hidden computational cost)
\end{itemize}

\missingfigure{Fig9}

If $\Delta T$ is very small (tends to zero), we squeeze all the poles very closed to $(1,0)$. We need very high numerical precision (use a lot of digits) to avoid instability.

Rule of thumb of control engineers: $f_S$ is between 10 and 20 times the system bandwidth we are interested in.

\missingfigure{Fig10}

\begin{remark}[Another way of managing the choice of $\Delta T$ w.r.t. the aliasing problem]
    \missingfigure{Fig11}

    The classical way to deal with aliasing is to use anti-alias analog filters

    \missingfigure{Fig12}

    For example, if A/D is at $1KHz$ ($\Delta T = 1ms$)
    \missingfigure{Fig13}

    \paragraph{Full Digital Approach} without analog anti-alias filter

    \missingfigure{Fig14}
\end{remark}
