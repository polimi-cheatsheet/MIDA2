\newlecture{Sergio Savaresi}{12/05/2020}

\subsection{Direct optimization of gain $K$}
\[
    S: \begin{cases}
        x(t+1) = 2x(t) \\
        y(t) = x(t) + v(t) \quad v \sim WN(0, 1)
    \end{cases}
\]

\paragraph{Problem} Compute the steady-state predictor of state $\hat{x}(t+1|t)$.

We use 2 different methods:
\begin{enumerate}
    \item Direct optimization
    \item K.F. theory
\end{enumerate}

\begin{rem}
    The system is not stable and the state equation is noise-free.
\end{rem}

\paragraph{First mehtod} Direct solution

Let's start from the standard observer structure
\[
    \begin{cases}
        \hat{x}(t+1|t) = 2\hat{x}(t|t-1) + K(y(t) - \hat{y}(t|t-1)) \\
        \hat{y}(t|t-1) = \hat{x}(t|t-1)
    \end{cases}
\]

Find the optimal $K$ by minimizing the variance of the state prediction error $\text{var}\left[x(t)-\hat{x}(t|t-1)\right]$.

Write the state prediction error expression:
\begin{align*}
    x(t+1) - \hat{x}(t+1|t) &= 2x(t) - \left[ 2\hat{x}(t|t-1) + K (y(t) - \hat{y}(t|t-1)) \right] =\\
    &= 2x(t) - 2\hat{x}(t|t-1) - K(x(t) + v(t) - \hat{x}(t|t-1)) = \\
    &= (2-K)(x(t) - \hat{x}(t|t-1)) - Kv(t)
\end{align*}

\paragraph{Definition} $\eta(t) = x(t) - \hat{x}(t|t-1)$

The dynamic equation of the state prediction error is:
\[
    \eta(t+1) = (2-K)\eta(t) - Kv(t) \qquad v \sim WN(0,1)
\]
It is an AR(1) process, in canonical form:
\[
    \eta(t) = \frac{1}{1-(2-K)z^{-1}}e(t) \qquad e(t) = -Kv(t) \qquad e \sim WN(0, K^2)
\]

It's easy to find the variance of $\eta(t)$:
\[
    \gamma_\eta(0) = \text{var}\left[\eta(t)\right] = \frac{K^2}{1-(2-K)^2}
\]

By minimizing this function with respect to $K$:
\[
    \frac{\partial \text{var} \left[\eta(t)\right]}{\partial K} = 0 \quad\Rightarrow\quad \begin{cases}
        K_1 = 0 & \text{(no need of feedback correction)} \\
        K_2 = \frac{3}{2}
    \end{cases}
\]

Both are acceptable solutions.

\paragraph{Second method} Using K.F. theory

From the system $S$: $F=2$, $H=1$, $V_1 = 0$ so $\Gamma = 0$, $V_2 = 1$, $V_{12} = 0$.

Theorem 1 requires that $V_{12} = 0$ (ok) and $F$ is asymptotically stable (not ok).

Theorem 2 requires that $V_{12} = 0$ (ok), $(F, \Gamma)$ not reachable (not ok), $(F, H)$ is observable (ok).

We cannot skip the analysis of D.R.E.

\[
    \text{D.R.E.} = 4P(t) - \frac{(2P(t))^2}{P(t)+1} \ldots
\]
\[
    P(t+1) = \frac{4P(t)}{P(t) + 1}
\]

Next, find the corresponding the A.R.E. solutions.
\[
    \overline{P} = \frac{4\overline{P}}{\overline{P}+1} \quad\Rightarrow\quad \begin{cases}
        \overline{P}_1 = 0 \\
        \overline{P}_2 = 3
    \end{cases}
    \quad\Rightarrow\quad \begin{cases}
        K_1 = 0 \\
        K_2 = \frac{3}{2}
    \end{cases}
\]

We need to make the D.R.E. convergence analysis.
\begin{figure}[H]
    \centering
    \begin{tikzpicture}[node distance=2.5cm,auto,>=latex']
        \fill [pattern=flexible hatch, pattern color=gray!40] (-1.5,-1.5) -- (5.5,-1.5) -- (5.5,0) -- (0,0) -- (0,4.9) -- (-1.5,4.9) -- (-1.5,-2);
        \draw[->] (-1.5,0) -- (6,0) node[below] {$P(t)$};
        \draw[->] (0,-1.5) -- (0,5) node[above] {$P(t+1)$};
        \draw[domain=-0.2:5,smooth,variable=\x,red] plot ({\x},{5*\x/(\x+1)});

        \draw[dashed] (-0.3,-1.5) -- (-0.3,4.9);
        \node[above left] at (-0.3,0) {$-1$};
        \draw[dashed] (-1.5,4.5) -- (5.5,4.5);
        \node[above right] at (0,4.5) {$4$};
        \node[below] at (4,0) {$3$};
        \draw[dotted] (4,0) -- (4,4);

        \draw (-1.5,-1.5) -- (5,5) node[left] {$P(t+1) = P(t)$};

        \draw[->,dotted] (1,0) -- (1,2.5);
        \draw[->,dotted] (1,2.5) -- (2.5,2.5);
        \draw[->,dotted] (2.5,2.5) -- (2.5,3.6);
        \draw[->,dotted] (2.5,3.6) -- (3.6,3.6);
        \draw[->,dotted] (3.6,3.6) -- (3.6,3.9);
        \draw[mark=*,red] plot coordinates {(4,4)} node[black,below right] {$\overline{P}_2$};
        \draw[mark=*,red] plot coordinates {(0,0)} node[black,below right] {$\overline{P}_1$};
        \draw[mark=*,blue] plot coordinates {(1,2.5)};
        \draw[mark=*,blue] plot coordinates {(2.5,3.6)};
        \draw[mark=*,blue] plot coordinates {(3.6,3.9)};

        \draw[->,red] (1.3,3.4) arc[radius=5,start angle=120,end angle=95];
        \node[red,rotate=19] at (2,3.9) {\tiny converges};
    \end{tikzpicture}
\end{figure}

If we start from $P_0=0$ (no uncertainty in the knowledge of $x(1)$) it converges into $\overline{P}_1=0$, the state equation is noise-free and we don't need feedback and the prediction is perfect (open-loop solution).
This is a feasible but ideal situation.

The standard solution (no perfect initial condition) is when $K=\frac{3}{2}$ (real closed-loop solution).

\chapter{Software-sensing with Black-Box Methods}

In chapter 3 we have seen classical technology of software-sensing based on Kalman Filter:
\begin{figure}[H]
    \centering
    \begin{tikzpicture}[node distance=2.5cm,auto,>=latex']
        \node[left] at (0,4) (u) {$u(t)$};
        \node[int] at (2,4) (sys) {system};
        \node[int] at (2,2.3) (k) {$K$};
        \node[sum] at (4,2.3) (sum) {};
        \node[right] at (6,4) (y) {$y(t)$};
        \node[int] at (2,1) (model) {model};
        \node[above] at (2,5) (dist) {disturbances};

        \draw[->] (dist) -- (sys);
        \draw[->] (u) -- (sys);
        \draw[->] (sys) -- (y);
        \draw[<-,red,line width=0.4mm] (sum) -- (4,4) node[pos=0.2] {+};
        \draw[->] (sum) -- (k);
        \draw[->,red,line width=0.4mm] (1,4) |- (model);
        \draw[->] (k) -- (model);
        \draw[->] (model) -| (sum) node[pos=0.8] {-};
        \draw[->,red,line width=0.4mm,transform canvas={yshift=-0.2cm}] (model) -- (6,1) node[right] {$\hat{x}(t)$};
        \draw[dotted] (0,0) rectangle (5,3) node[right] {Kalman Filter};
    \end{tikzpicture}
\end{figure}

Main features of this approach:
\begin{itemize}
    \item A (white-box/physical) model must be available.
    \item No need (in principle) of a training dataset including measurements of the state to be estimated.
    \item It is a feedback estimation algorithm (feedback correction of the model using estimated output error).
    \item Constructive method (non-parametric, no optimization involved).
    \item Can be used (in principle) to estimate states which are impossible to be measured (also at prototyping/training/design stage).
\end{itemize}

Are there other classes of software-sensing techniques? Yes, black-box approaches with \emph{learning}/\emph{training} from data.

In this chapter we see them focusing on the architecture (we do not need new algorithms, just use something we have already studied).

\section{Linear Time Invariant Systems}

Let's consider a simplified case (SISO system with one state) to understand the approach.
\[
    S: \begin{cases}
        x(t+1) = ax(t) + bu(t) + v_1(t) \\
        y(t) = cx(t) + v_2(t)
    \end{cases}
\]

\paragraph{Problem} Estimation of $\hat{x}(t)$ from measured signals $u(t)$ and $y(t)$.

We can start from the observer K.F. architecture:
\begin{figure}[H]
    \centering
    \begin{tikzpicture}[node distance=2.5cm,auto,>=latex']
        \node [left] (u) at (-4,4) {$u(t)$};
        \node [int] (b) at (-3,4) {$b$};
        \node [int] (z1) at (1,4) {$z^{-1}$};
        \node [int] (F1) at (1,3) {$a$};
        \node [int] (K) at (1,2) {$K$};
        \node [int] (z2) at (1,1) {$z^{-1}$};
        \node [int] (F2) at (1,0) {$a$};

        \node [int] (H1) at (4,4) {$c$};
        \node [int] (H2) at (4,1) {$c$};

        \node [sum] (sum1) at (6,4) {};
        \node [sum] (sum2) at (7,2) {};
        \node [sum] (sum3) at (-1.5,4) {};
        \node [sum] (sum4) at (-1.5,1) {};

        \node (v1) at (-1.5,5) {$v_1(t)$};
        \node (v2) at (6,5) {$v_2(t)$};
        \node[right] (y) at (8,4) {$y(t)$};
        \node[right] (yhat) at (8,1) {$\hat{y}(t|t-1)$};
        \node[right] (xhat) at (8,0) {$\hat{x}(t|t-1)$};

        \draw[->] (u) -- (b);
        \draw[->] (b) -- (sum3);
        \draw[->] (-2.2,4) |- (sum4);
        \draw[->] (v1) -- (sum3);
        \draw[->] (sum3) -- (z1) node[pos=0.5] {$x(t+1)$};
        \draw[->] (F1) -| (sum3);
        \draw[->] (z1) -- (H1) node[pos=0.5] {$x(t)$};
        \draw[->] (H1) -- (sum1);
        \draw[->] (v2) -- (sum1);
        \draw[->] (2.5,4) |- (F1);
        \draw[->] (K) -| (sum4);
        \draw[->,blue,line width=0.5mm] (sum4) -- (z2) node[pos=0.5,black] {$\hat{x}(t+1|t)$};
        \draw[<-] (sum2) -- (7,4) node[pos=0.2] {$+$};
        \draw[->] (sum1) -- (y);
        \draw[<-] (K) -- (sum2) node[pos=0.5,black] {$e(t)$};
        \draw[blue,line width=0.5mm] (z2) -- (2.5,1) node[pos=0.8,black] {$\hat{x}(t|t-1)$};
        \draw[->] (2.5,1) -- (H2) {};
        \draw[->,blue,line width=0.5mm] (F2) -| (sum4);
        \draw[->,blue,line width=0.5mm] (2.5,1) |- (F2);
        \draw[->] (H2) |- (yhat);
        \draw[->] (7,1) -- (sum2) node[pos=0.8] {$-$};
        \draw[->] (2.5,0) -- (xhat);

        \node[blue] at (-0.5,0.5) {$\frac{z^{-1}}{1-az^{-1}}$};
    \end{tikzpicture}
\end{figure}

Let's find the relationship between $u(t) \rightarrow \hat{x}(t)$ and $y(t) \rightarrow \hat{x}(t)$.
\begin{align*}
    \hat{x}(t) &= \frac{b {\color{blue}\frac{z^{-1}}{1-az^{-1}}}}{1+Kc{\color{blue}\frac{z^{-1}}{1-az^{-1}}}} u(t) + \frac{ K{\color{blue}\frac{z^{-1}}{1-az^{-1}}} }{1+Kc{\color{blue}\frac{z^{-1}}{1-az^{-1}}}} y(t) \\
    &= \frac{b}{1+(Kc-a)z^{-1}}u(t-1) + \frac{K}{1+(Kc-a)z^{-1}} y(t-1)
\end{align*}

We can make a Black-Box estimation of these two transfer functions from data.
