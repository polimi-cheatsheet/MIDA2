\newlecture{Sergio Savaresi}{07/05/2020}

To answer those questions we need two fundamental theorems (K.F. asymptotic theorems).
They provide \emph{sufficient} conditions only.

\paragraph{First asymptotic theorem}

Assumption: $V_{12} = 0$ and the system is asymptotically stable (all eigenvalues of $F$ are strictly inside the unit circle).
Then:
\begin{itemize}
    \item A.R.E. has one and only one semi-definite positive solution: $\overline{P} \ge 0$
    \item D.R.E. converges to $\overline{P}$, $\forall P_0 \ge 0$
    \item The corresponding $\overline{K}$ is s.t. the K.F. is asymptotically stable
\end{itemize}

\begin{recall}
    \paragraph{Observability} the pair $(F, H)$ is observable if and only if
    \[
        O = \begin{bmatrix}
            H \\
            HF \\
            \vdots \\
            HF^{n-1}
        \end{bmatrix}
        \qquad
        \text{is max rank}
    \]

    \paragraph{Controllability from the noise}

    \begin{align*}
        x(t+1) = Fx(t) + Gu(t) + v_1(t) \qquad v_1(t) \sim WN(0, V_1)
    \end{align*}

    We need controllability from $v_1(t)$ (and don't care about $u(t)$).

    \begin{align*}
        x(t+1) = Fx(t) + v_1(t)
    \end{align*}

    It's always possible to factorize $V_1 = \Gamma\cdot\Gamma^T$ rewriting
    \[
        x(t+1) = Fx(t) + \Gamma\omega(t) \qquad \omega(t) \sim WN(0, I)
    \]

    We can say that the state $x$ is controllable/reachable from the input noise $v_1(t)$ if and only if:
    \[
        R = \begin{bmatrix}
            \Gamma & F\Gamma & \cdots & F^{n-1}\Gamma
        \end{bmatrix}
    \]
\end{recall}

\paragraph{Second asymptotic theorem}

Assumption: $V_{12} = 0$, $(F, H)$ is observable and $(F, \Gamma)$ is controllable.
Then:
\begin{itemize}
    \item A.R.E. has one and only one definite positive solution $\overline{P} > 0$
    \item D.R.E. converges to $\overline{P}$, $\forall P_0 \ge 0$
    \item The corresponding $\overline{K}$ is such that the K.F. is asymptotically stable
\end{itemize}

There theorems are very useful in practice because we can fully avoid the (very difficult) convergence analysis of D.R.E.

\textbf{Note} that the two theorems provide only sufficient conditions.

\begin{exercise}
    \[
        S: \begin{cases}
            x(t+1) = \frac{1}{2}x(t) + v_1(t) \quad& v_1 \sim WN(0, \frac{19}{20}) \\
            y(t) = 2x(t) + v_2(t) \quad& v_2 \sim WN(0, 1)
        \end{cases}
        \qquad
        v_1 \perp v_2
    \]

    \paragraph{Question} Find (if exists) the steady state (asymp.) K.F. $\hat{x}(t+1|t)$ and $\hat{x}(t|t)$.

    \[
        n = 1 \qquad F = \frac{1}{2} \qquad G = 0 \qquad H = 2 \qquad V_1 = \frac{19}{20} \qquad V_2 = 1 \qquad V_{12} = 0
    \]

    Since $V_{12} = 0$ we can try to use the asymptotic theorems.

    \paragraph{First step} Compute the D.R.E.
    \begin{align*}
        P(t+1) &= \left( FP(t)F^T+V_1 \right) - \left( FP(t)H^T+V_{12} \right) \left( HP(t)H^T+V_2 \right)^{-1}\left( FP(t)H^T+V_{12} \right)^T \\
        &= \frac{1}{4}P(t) + \frac{19}{20} - \frac{ \left(\frac{1}{\cancel{2}}P(t)\cancel{2}\right)^2 }{4P(t)+1} \\
        &= \frac{\cancel{P(t)^2} + \frac{1}{4}P(t) + \frac{19}{5} P(t) + \frac{19}{20} - \cancel{P(t)^2} }{4P(t) + 1}
    \end{align*}

    \textbf{Note} The second order terms must cancel out.

    \[
        P(t+1) = \frac{81P(t) + 19}{80P(t)+20}
    \]

    \paragraph{Second step} Compute and solve the A.R.E.

    \[
        \overline{P} = \frac{81\overline{P} + 19}{80\overline{P}+20} \quad \implies \quad 80\overline{P}^2 + 20\overline{P}-81\overline{P}-19 = 0
    \]
    \[
        \overline{P}_1 = 1 \qquad \cancel{\overline{P}_2 = -\frac{19}{80}} < 0
    \]


    $\overline{P}=1$ is the only definite positive solution of A.R.E.

    \paragraph{Question} Does D.R.E. converges to $\overline{P}=1$, $\forall P_0 \ge 0$?

    There are 2 methods for addressing this question:
    \begin{itemize}
        \item Direct analysis of D.R.E.
        \item Using asymptotic theorems
    \end{itemize}

    \paragraph{First methods} Direct analysis of D.R.E.

    \begin{align*}
        P(t+1) = f(P(t)) \qquad \text{we need to plot $f(\cdot)$ in the $P(t)$ -- $P(t+1)$ plane}
    \end{align*}

    \[
        \overline{P} = \frac{81\overline{P} + 19}{80\overline{P}+20}
        \qquad
        \begin{cases}
            \text{vertical asy. value} & P(t) = -\frac{20}{80} = -\frac{1}{4} \\
            \text{horizontal asy. value} & P(t) = \frac{81}{80}
        \end{cases}
    \]

    \begin{figure}[H]
        \centering
        \begin{tikzpicture}[node distance=2.5cm,auto,>=latex']
            \fill [pattern=flexible hatch, pattern color=gray!40] (-2.5,-2) -- (5.5,-2) -- (5.5,0) -- (0,0) -- (0,4.9) -- (-2.5,4.9) -- (-2.5,-2);
            \draw[->] (-2.5,0) -- (6,0) node[below] {$P(t)$};
            \draw[->] (0,-2) -- (0,5) node[above] {$P(t+1)$};
            \draw[domain=-1.25:5,smooth,variable=\x,red] plot ({\x},{5*(\x+1)/(\x+2)});

            \draw[dashed] (-1.5,-2) -- (-1.5,4.9);
            \node[above left] at (-1.5,0) {$-\frac{1}{4}$};
            \draw[dashed] (-2.5,4.5) -- (5.5,4.5);
            \node[below left] at (0,4.5) {$\frac{81}{80}$};
            \node[above left] at (0,2.5) {$\frac{19}{20}$};

            \draw (-1.8,-1.8) -- (5,5) node[left] {$P(t+1) = P(t)$};

            \draw[->,dotted] (1,0) -- (1,3.3);
            \draw[->,dotted] (1,3.3) -- (3.3,3.3);
            \draw[->,dotted] (3.3,3.3) -- (3.3,4.0);
            \draw[->,dotted] (3.3,4.0) -- (4.0,4.0);
            \draw[->,dotted] (4.0,4.0) -- (4.0,4.2);
            \draw[mark=*,red] plot coordinates {(4.2,4.2)} node[black,below] {$\overline{P}_1$};
            \draw[mark=*,red] plot coordinates {(-1.2,-1.2)} node[black,below right] {$\overline{P}_2$};
            \draw[mark=*,blue] plot coordinates {(1,3.35)} node[black,above] {\tiny$P(2)$};
            \draw[mark=*,blue] plot coordinates {(3.3,4.05)} node[black,above] {\tiny$P(3)$};
            \draw[mark=*,blue] plot coordinates {(4,4.2)};
            \node[below] at (1,0) {\tiny$P(1) = P_0 \ge 0$};

            \draw[->,red] (1.3,3.9) arc[radius=5,start angle=115,end angle=95];
            \node[red,rotate=15] at (2,4.3) {\tiny converges};
        \end{tikzpicture}
    \end{figure}

    By direct analysis/inspection of D.R.E. dynamics we can conclude that $\forall P_0\ge 0$, D.R.E. always converges to $\overline{P}_1=1$.

    If $n=1$ the direct inspection is feasible, but it's very difficult for $n\ge2$.

    \paragraph{Second method} Use theorems

    \[
        V_{12} = 0 \qquad F = \frac{1}{2} \text{ ($S$ is stable)} \quad \implies \quad \text{First theorem is fulfilled}
    \]

    The observability matrix of $\{F, H\}$ is $O=\begin{bmatrix} 2 \end{bmatrix}$ with full $\rank O = 1$, the system is fully observable.

    Controllability from noise $v_1(t)$
    \[
        V_1 = \frac{19}{20} \qquad \Gamma=\sqrt{\frac{19}{20}}
    \]
    The controllability matrix of $\{F, \Gamma\}$ is $R = \begin{bmatrix} \sqrt{\frac{19}{20}} \end{bmatrix}$ with full $\rank R = 1$, the system is fully controllable from noise.

    Both theorems are fulfilled, so A.R.E. has one and only one solution $\overline{P} > 0$, D.R.E. converges to $\overline{P}$, $\forall P_0\ge0$ and $\overline{K}$ makes the K.F. asymptotically stable.

    \paragraph{Third step} Compute $\overline{K}$
    \[
        \overline{K} = \left(F\overline{P}H^T + V_{12}\right)\left(H\overline{P}H^T+V_2\right) = \left(\frac{1}{2}\cdot 1 \cdot 2 + 0\right) \left(2\cdot 1 \cdot 2 + 1\right)^{-1} = \frac{1}{5}
    \]

    Double-check the asymptotical stability of K.F.
    \[
        F - \overline{K}H = \frac{1}{2} - \frac{1}{5}\cdot 2 = \frac{5-4}{10} = \frac{1}{10}
    \]

    \begin{figure}[H]
        \centering
        \begin{tikzpicture}[node distance=2.5cm,auto,>=latex']
            \node [int] (z1) at (1,4) {$z^{-1}$};
            \node [int] (F1) at (1,3) {$\frac{1}{2}$};
            \node [int] (K) at (1,2) {$\frac{1}{5}$};
            \node [int] (z2) at (1,1) {$z^{-1}$};
            \node [int] (F2) at (1,0) {$\frac{1}{2}$};

            \node [int] (H1) at (4,4) {$2$};
            \node [int] (H2) at (4,1) {$2$};

            \node [sum] (sum1) at (6,4) {};
            \node [sum] (sum2) at (7,2) {};
            \node [sum] (sum3) at (-1.5,4) {};
            \node [sum] (sum4) at (-1.5,1) {};

            \node (v1) at (-1.5,5) {$v_1(t)$};
            \node (v2) at (6,5) {$v_2(t)$};
            \node[right] (y) at (8,4) {$y(t)$};
            \node[right] (yhat) at (8,1) {$\hat{y}(t|t-1)$};
            \node[right] (xhat) at (8,0) {$\hat{x}(t|t-1)$};

            \draw[->] (v1) -- (sum3);
            \draw[->] (sum3) -- (z1) node[pos=0.5] {$x(t+1)$};
            \draw[->] (F1) -| (sum3);
            \draw[->] (z1) -- (H1) node[pos=0.5] {$x(t)$};
            \draw[->] (H1) -- (sum1);
            \draw[->] (v2) -- (sum1);
            \draw[->] (2.5,4) |- (F1);
            \draw[->] (K) -| (sum4);
            \draw[->] (sum4) -- (z2) node[pos=0.5] {$\hat{x}(t+1|t)$};
            \draw[->] (7,4) -- (sum2) node[pos=0.8] {$+$};
            \draw[->] (sum1) -- (y);
            \draw[<-] (K) -- (sum2) node[pos=0.5,black] {$e(t)$};
            \draw[->] (z2) -- (H2) node[pos=0.5] {$\hat{x}(t|t-1)$};
            \draw[->] (F2) -| (sum4);
            \draw[->] (2.5,1) |- (F2);
            \draw[->] (H2) |- (yhat);
            \draw[->] (7,1) -- (sum2) node[pos=0.8] {$-$};
            \draw[->] (F2) -- (xhat);
        \end{tikzpicture}
    \end{figure}

    \paragraph{Question} Find T.F. from $y(t)$ to $\hat{x}(t|t-1)$

    \textbf{Recall} T.F. from block schemes of feedback systems.

    \begin{figure}[H]
        \centering
        \begin{tikzpicture}[node distance=2.5cm,auto,>=latex']
            \node[int] (G1) at (0,1) {$G_1(z)$};
            \node[int] (G2) at (0,0) {$G_2(z)$};
            \node[sum] (sum) at (-1.5,1) {};
            \node (u) at (-3,1) {$u(t)$};
            \node (y) at (2.5,1) {$y(t)$};

            \draw[->] (u) -- (sum) node[pos=0.8] {$+$};
            \draw[->] (sum) -- (G1);
            \draw[->] (G1) -- (y);
            \draw[->] (G2) -| (sum) node[pos=0.8] {$\mp$};
            \draw[->] (1.5,1) |- (G2);
        \end{tikzpicture}
    \end{figure}

    \[
        y(t) = G_1(z) \left(u(t) \mp G_2(z)y(t)\right) \quad \implies \quad y(t) = \frac{G_1(z)}{1 \pm G_1(z)G_2(z)}u(t)
    \]

    The K.F. is composed of 2 nested loops.

    \begin{figure}[H]
        \centering
        \begin{tikzpicture}[node distance=2.5cm,auto,>=latex']
            \node [int] (z1) at (1,4) {$z^{-1}$};
            \node [int] (F1) at (1,3) {$\frac{1}{2}$};
            \node [int] (K) at (1,2) {$\frac{1}{5}$};
            \node [int] (z2) at (1,1) {$z^{-1}$};
            \node [int] (F2) at (1,0) {$\frac{1}{2}$};

            \node [int] (H1) at (4,4) {$2$};
            \node [int] (H2) at (4,1) {$2$};

            \node [sum] (sum1) at (6,4) {};
            \node [sum] (sum2) at (7,2) {};
            \node [sum] (sum3) at (-1.5,4) {};
            \node [sum] (sum4) at (-1.5,1) {};

            \node (v1) at (-1.5,5) {$v_1(t)$};
            \node (v2) at (6,5) {$v_2(t)$};
            \node[right] (y) at (8,4) {$y(t)$};
            \node[right] (yhat) at (8,1) {$\hat{y}(t|t-1)$};
            \node[right] (xhat) at (8,0) {$\hat{x}(t|t-1)$};

            \draw[->] (v1) -- (sum3);
            \draw[->] (sum3) -- (z1) node[pos=0.5] {$x(t+1)$};
            \draw[->] (F1) -| (sum3);
            \draw[->] (z1) -- (H1) node[pos=0.5] {$x(t)$};
            \draw[->] (H1) -- (sum1);
            \draw[->] (v2) -- (sum1);
            \draw[->] (2.5,4) |- (F1);
            \draw[->,red,line width=0.5mm] (K) -| (sum4);
            \draw[->,red,line width=0.5mm] (sum4) -- (z2) node[pos=0.5,black] {$\hat{x}(t+1|t)$};
            \draw[->] (7,4) -- (sum2) node[pos=0.8] {$+$};
            \draw[->] (sum1) -- (y);
            \draw[<-] (K) -- (sum2) node[pos=0.5,black] {$e(t)$};
            \draw[->,red,line width=0.5mm] (z2) -- (H2) node[pos=0.5,black] {$\hat{x}(t|t-1)$};
            \draw[->,red,line width=0.5mm] (F2) -| (sum4);
            \draw[->,red,line width=0.5mm] (2.5,1) |- (F2);
            \draw[->] (H2) |- (yhat);
            \draw[->] (7,1) -- (sum2) node[pos=0.8] {$-$};
            \draw[->] (2.5,0) -- (xhat);

            \node[red] at (-0.5,0.5) {$\frac{z^{-1}}{1-\frac{1}{2}z^{-1}}$};
        \end{tikzpicture}
    \end{figure}

    Predictor of state:
    \[
        \hat{x}(t|t-1) = \frac{ \frac{1}{5} \frac{z^{-1}}{1-\frac{1}{2}z^{-1}} }{ 1 + \frac{1}{5} \frac{z^{-1}}{1-\frac{1}{2}z^{-1}} 2 }y(t) = \frac{ \frac{1}{5}z^{-1} }{1-\frac{1}{10}z^{-1}}y(t)
    \]

    Predictor of output:
    \[
        \hat{y}(t|t-1) = H\hat{x}(t|t-1) = \frac{2}{5} \frac{ 1 }{1-\frac{1}{10}z^{-1}}y(t-1)
    \]

    Filter of state:
    \[
        \hat{x}(t|t) = F^{-1}\hat{x}(t+1|t) = \frac{2}{5} \frac{ 1 }{1-\frac{1}{10}z^{-1}}y(t)
    \]
\end{exercise}
